\documentclass{beamer}
\usetheme[style=plain]{uu}

\usepackage{graphicx}
\usepackage{mathtools}
\usepackage{amssymb}
\usepackage{hyperref}
\usepackage{color}
\usepackage{float}

%% unicode support, for text and code :)
\usepackage[utf8x]{inputenc}
\usepackage{ucs}
\usepackage{autofe}


%% Haskell syntax highlighting and code font
\usepackage{minted}
\usepackage{fancyvrb}
\usepackage{inconsolata}
\usemintedstyle{default}

\newmint{haskell}{mathescape,fontfamily=tt}
\newmintedfile{haskell}{xleftmargin=20pt,mathescape,fontfamily=tt}

\newminted{haskell}{xleftmargin=20pt,gobble=8,mathescape,fontfamily=tt}
\newmint{coq}{mathescape,fontfamily=tt}

\newmintedfile{coq}{xleftmargin=20pt,mathescape,fontfamily=tt}
\newminted{coq}{gxleftmargin=20pt,obble=8,mathescape,fontfamily=tt}



%% Metainformation
%% PDF stuff
\usepackage{datetime}
\usepackage{ifpdf}
\ifpdf
\pdfinfo{
    /Author (Joao Paulo Pizani Flor)
    /Title (Comparing functional EDSLs for hardware description)
    /Keywords (Hardware verification, Functional Programming, Hardware design, Dependently-typed programming, Coq, Lava, ForSyDe, Haskell, Coquet)
    /CreationDate (D:\pdfdate)
}
\fi

\title[Comparing functional EDSLs for hardware description]{Comparing functional Embedded Domain-Specific Languages for hardware description}

\date{February 13th, 2014}

\author[Pizani Flor]
{
    João Paulo Pizani Flor
}

\institute[Utrecht University]
{
    Department of Information and Computing Sciences,
    Utrecht University
}

\subject{Function Programming, Hardware verification, Haskell, Coq, Lava, Coquet, ForSyDe}




\begin{document}

%% The document itself
    \begin{frame}
        \titlepage
    \end{frame}

    \begin{frame}
        \frametitle{Table of Contents}
        \tableofcontents
    \end{frame}



    \section{Introduction}
    \label{sec:introduction}
        \frame{\sectionpage}

        \subsection{Hardware design}
        \label{subsec:hardware-design}
            \begin{frame}
                \frametitle{Hardware design}
            \end{frame}

        \subsection{Domain-Specific Languages}
        \label{subsec:domain-specific-languages}
            % deep-embedded / shallow embedded
            \begin{frame}
                \frametitle{Domain-Specific Languages}

                \par{A computer language (turing-complete or \emph{not}) targeting a \emph{specific application domain.}}
                \par{\textbf{Example DSLs:}}
                \begin{itemize}
                    \item SQL (database queries)
                    \item CSS (document formatting)
                    \item MATLAB (Matrix programming)
                    \item VHDL (Hardware description)
                \end{itemize}

                \pause

                \par{A DSL can also be \emph{embedded} in a general-purpose language.}
                \par{\textbf{Example EDSLs:}}
                \begin{itemize}
                    \item Boost.Proto (C++ / parser combinators)
                    \item Diagrams (Haskell / programmatic drawing)
                    \item Parsec (Haskell / parser combinators)
                \end{itemize}
            \end{frame}

            \begin{frame}[fragile]
                \frametitle{Example of an EDSL: \texttt{Parsec}}

                A simple parser for a "Game of Life"-like input format:
                \haskellfile{code/parsec-example.hs}
\end{frame}


        \subsection{Hardware EDSLs}
        \label{subsec:hardware-edsls}
            \begin{frame}
                \frametitle{Hardware EDSLs}
                An EDSL used for hardware design-related tasks. Can encompass:

                \begin{itemize}
                    \item Modelling / description
                    \item Simulation (validation)
                    \item Formal verification
                    \item Synthesis to other (lower-level) languages
                \end{itemize}
            \end{frame}

            \begin{frame}
                \frametitle{Example of a hardware EDSL}

                Some Lava code\ldots
            \end{frame}


    \section{Analyzed EDSLs}
    \label{sec:analyzed-edsls}
        \frame{\sectionpage}

        \subsection{Choice criteria}
        \label{subsec:edsls-choice-criteria}
            \begin{frame}
                \frametitle{Choice criteria}
            \end{frame}

        \subsection{Chosen EDSLs}
        \label{subsec:chosen-edsls}
            \begin{frame}
                \frametitle{Chosen EDSLs}
                \begin{itemize}
                    \item Lava
                    \item ForSyDe
                    \item Coquet
                \end{itemize}
            \end{frame}

        \subsection{Evaluation criteria}
        \label{subsec:evaluation-criteria}
            \begin{frame}
                \frametitle{Evaluation criteria}
            \end{frame}


    \section{Modeled Circuits}
    \label{sec:modeled-circuits}
        \frame{\sectionpage}

        \subsection{Choice criteria}
        \label{subsec:circuit-choice-criteria}
            \begin{frame}
                \frametitle{Choice criteria}
            \end{frame}

        \subsection{ALU}
        \label{subsec:alu}
            \begin{frame}
                \frametitle{ALU}
            \end{frame}

        \subsection{Memory bank}
        \label{subsec:memory-bank}
            \begin{frame}
                \frametitle{Memory bank}
            \end{frame}

        \subsection{CPU}
        \label{subsec:cpu}
            \begin{frame}
                \frametitle{CPU}
            \end{frame}


    \section{Analysis of the EDSLs}
    \label{sec:analysis-of-the-edsls}
        \frame{\sectionpage}

        \subsection{Lava}
        \label{subsec:lava}
            \begin{frame}
                \frametitle{Lava}
            \end{frame}

        \subsection{ForSyDe}
        \label{subsec:forsyde}
            \begin{frame}
                \frametitle{ForSyDe}
            \end{frame}

        \subsection{Coquet}
        \label{subsec:coquet}
            \begin{frame}
                \frametitle{Coquet}
            \end{frame}


    \section{Conclusions}
    \label{sec:conclusions}
        \frame{\sectionpage}

        \begin{frame}[plain]
            \begin{center}
                \par{\Huge{Thank you!}}
                \vspace{2.0cm}
                \par{\Huge{Questions?}}
            \end{center}
        \end{frame}


\end{document}
