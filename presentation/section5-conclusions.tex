\section{Conclusions}
\label{sec:conclusions}
    \frame{\sectionpage}

    \begin{frame}
        \frametitle{Results}
        \par{Summary of our findings, by aspect:}
        \vspace{0.2cm}
        \begin{itemize}
            \item \textbf{Depth of embedding}
                \begin{itemize}
                    \item Lava: deep-embedded, recursion and sharing through host
                    \item ForSyDe: \emph{both} shallow and deep-embedded signals
                    \item Coquet: the \emph{deepest} of all, circuit \emph{structure in the AST}
                \end{itemize}
            \item \textbf{Simulation}
                \begin{itemize}
                    \item Lava: straightforward, but not easily \emph{automated}
                    \item ForSyDe: easy in both embedding depths
                    \item Coquet: one of the \emph{example} interpretations, not sequential
                \end{itemize}
            \item \textbf{Verification}
                \begin{itemize}
                    \item Lava: \emph{safety properties} through external SAT solver
                    \item ForSyDe: no capabilities of verification whatsoever
                    \item Coquet: Interactive theorem proving, verifies \emph{families} of circuits.
                \end{itemize}
        \end{itemize}

        \note{
            \begin{itemize}
                \item We focused on ForSyDe's deep model
                \item ForSyDe diff. libraries promised for shallow and deep embedding
                \item Coquet avois observable sharing issues by using a nameless setting, and looping only with a specific constructor
                \item Coquet Simulation not commented in the paper, also case for Loop simply omitted
                \item Lava safety property: boolean value true for all input combs. CNF. FIXED SIZE.
            \end{itemize}
        }
    \end{frame}

    \begin{frame}
        \frametitle{Results}
        \par{Summary of our findings, by aspect:}
        \vspace{0.2cm}
        \begin{itemize}
            \item \textbf{Genericity}
                \begin{itemize}
                    \item Lava: \emph{families} of circuits, with extra arguments
                    \item ForSyDe: weak genericity, monomorphic types in \texttt{ProcFun}'s
                    \item Coquet: similar approach to Lava
                \end{itemize}
            \item \textbf{Tool integration}
                \begin{itemize}
                    \item Lava: \emph{flat} VHDL code (\texttt{Signal Bool}) and CNF formulas
                    \item ForSyDe: \emph{modular} VHDL code and GraphML files
                    \item Coquet: no circuit extraction whatsoever
                \end{itemize}
            \item \textbf{Extensibility}
                \begin{itemize}
                    \item Lava: no \emph{data} extensibility, high \emph{structural} extensibility
                    \item ForSyDe: possible to use custom \emph{enumerated} types
                    \item Coquet: flexible approach to data extensibility with meaning relation
                \end{itemize}
        \end{itemize}

        \note{
            \begin{itemize}
                \item Lava arguments diff. than (Signal a) are taken as parameters, not inputs
                \item ForSyDe most genericity you can do is define type synonyms (not really)
                \item It is POSSIBLE to generate VHDL output containing generic constructs
                \item Coquet has a gate-count as an example, so netlist generation isn't far-fetched
                \item Lava can only work with signals of Bool and Int (only Bool synthesizable)
                \item Lava can use the host language's ordinary higher-order functions as combinators
                \item Coquet has VERY WEAK \emph{structural} extensibility
            \end{itemize}
        }
    \end{frame}

    \begin{frame}
        \frametitle{Future work}
    \end{frame}


    \begin{frame}[plain]
        \begin{center}
            \par{\Huge{Thank you!}}
            \vspace{2.0cm}
            \par{\Huge{Questions?}}
        \end{center}
    \end{frame}
