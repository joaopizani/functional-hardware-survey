\section{Methodology}
\label{sec:methods}

    In this project, we compared a number of functional hardware EDSLs that we considered
    representative (more details on the choice of EDSLs further ahead). The comparison was performed
    on a number of \emph{aspects} for each EDSL, and the analysis was done by considering a
    \emph{sample set of circuits} used as case studies.

    We tried to model all circuits in all EDSLs considered, and as similarly as possible in each
    EDSL. To avoid using any of the analyzed EDSLs as ``base'' for analysis, we provide a neutral,
    behavioural description of the circuits.

    \subsection{The languages}
    \label{subsec:languages}
        The embedded Hardware Description Languages we decided to analyze are:

        \begin{description}
            \item[Lava] The Lava\cite{lava1998} language, developed initially at Chalmers University
            in Sweden.  Lava is deeply embedded in Haskell, and provides features such as netlist
            generation and circuit verification using SAT-solvers. There are several ``dialects'' of
            Lava available, and the one used for this project is considered the ``canonical'' one,
            originally developed at Chalmers.

            \item[ForSyDe] The Haskell ForSyDe library is an EDSL based on the ``Formal System
            Design'' approach\cite{forsyde1999}, developed at the swedish Royal Institute of
            Technology (KTH).  It offers both shallow and deep embeddings, and provides a
            significantly different approach to circuit modeling, using \emph{Template Haskell} to
            allow the designer to describe combinational functions with Haskell's own constructs.

            \item[Coquet] The Coquet\cite{coquet2011} EDSL differs from the other 2 mainly because
            it's embedded in a dependently-typed programming language (the Coq theorem prover).
            Coquet aims to allow the hardware designer to describe his circuits and then
            \emph{interactively} prove theorems about the behaviour of whole \emph{families} of
            circuits (using proofs by induction).
        \end{description}


    \subsection{The aspects evaluated}
    \label{subsec:aspects}

        For each of the hardware description EDSLs we experimented with, a number of \emph{aspects}
        were evaluated. The evaluated aspects do not necessarily make sense for \emph{all} EDSLs,
        therefore our presentation follows a language-centric approach, in which we expose the
        strengths and weaknesses of each EDSL concerning the applicable aspects.

        Without further ado, the following aspects are considered in the analysis:

        \begin{description}

            \item[Simulation] The capability of simulating circuits modeled in the EDSL (and the
            ease with which it can be performed). Simulation is understood in this context as
            \emph{functional} simulation, i.e, obtaining the outputs calculated by the circuit for
            certain input combinations.

            \item[Verification] The capability of verifying \emph{formal properties} concerning the
            behaviour of circuits (and the ease with which verification can be performed). The
            properties we are interested in are those which are \emph{universally quantified} over
            the circuit's inputs. As an example of such a property, we might have: \[ \forall a
            \forall b \forall \text{\textit{sel}} \left( \text{MUX}(a,b,\text{\textit{sel}}) = a
            \right) \vee \left( \text{MUX}(a,b,\text{\textit{sel}}) = b \right) \]

            \item[Genericity] Whether (and how well) the EDSL allows the modeling of \emph{generic}
            or \emph{parameterized} circuits. An example of a generic circuit is a multiplexer with
            2 n-bit inputs and 1 n-bit output, or a multiplexer with n 1-bit inputs and 1 1-bit
            output. Besides parametrization in the \emph{size} if inputs and outputs, we will also
            analyze whether the EDSL provides chances for parametrization on other functional and/or
            non-functional attributes.

            \item[Depth of embedding] Whether the EDSL models circuits with a \emph{shallow}
            embedding (using predicates or functions of the host language), a \emph{deep embedding}
            (in which circuits are members of a dedicated data type), or anything in between. The
            depth of embedding of an EDSL might have consequences for other aspects being analyzed.

            \item[Integration with other tools] How well does the EDSL allow for interaction with
            (getting input from / generating output for) other tools in the hardware design process.
            For example, synthesis tools for FPGAs or ASICs, timing analysis tools, model checkers,
            etc.

            \item[Extensibility] The extent to which the user can \emph{add} new interpretations,
            data types, and combinator forms to the language. For example, the user might want to
            model circuits that consume and produce custom datatypes, or might be interested in
            extracting \emph{metrics} from a circuit such as power consumption, number of elementary
            gates, etc.
        \end{description}
