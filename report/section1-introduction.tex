\section{Introduction}
\label{sec:intro}

    Hardware design has become a very complex activity. The size of circuits has increased, while
    low-level concerns (power consumption, error correction, parallelization, layout, etc.) have to
    be incorporated earlier and earlier in the design process. This breaks modularity and makes it
    harder to validate and verify the correctness of circuits.

    \acrodef{EDSL}{Embedded Domain-Specific Languages} In this context, researchers have been
    suggesting (since the 1980s) the usage of functional programming languages to model circuits.
    One particular line of research is to create \acp{EDSL} for hardware description based on
    existing functional programming languages, such as Haskell.

    There are a multitude of \acp{EDSL} for hardware description, but they vary wildly on a number
    of aspects: host language, level of abstraction, capabilities of simulation, formal
    verification, synthesis (generation of netlists) and integration with other tools, to name a
    few. All this variety can make the task of choosing a hardware EDSL for the task at hand
    daunting and time-consuming.

    The main goal of this experimentation project is to establish some order in this landscape, and
    to perform a practical analysis of some popular functional hardware \acp{EDSL}. By reading the
    materials produced in this project (circuit models, test cases, generated netlists, report), a
    hardware designer wishing to use a functional hardware \ac{EDSL} for his next design should gain
    some insight about the strengths and weaknesses of each language and have an easier time
    choosing one.

    As an additional result of this research, we intend to identify recent, cutting-edge
    developments in the Haskell language and its implementations from which the analyzed \acp{EDSL}
    could benefit. Also, we intend to discuss to which extent some shortcomings of the \acp{EDSL}
    could be overcome by having them hosted in a dependently-typed language.
