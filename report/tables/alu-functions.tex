\begin{table}[h]
    \begin{center}
        \begin{tabular}{ccccccc}
            \toprule
            \textbf{zx} & \textbf{nx} & \textbf{zy} &
            \textbf{ny} & \textbf{f} & \textbf{no} & \textbf{out=}
            \tabularnewline
            \midrule

            1  &  0  &  1  &  0  &  1  &  0  &   $0$  \\

            1  &  1  &  1  &  1  &  1  &  1  &   $1$  \\

            1  &  1  &  1  &  0  &  1  &  1  &   $-1$  \\

            0  &  0  &  1  &  1  &  0  &  0  &   $x$  \\

            1  &  1  &  0  &  0  &  0  &  0  &   $y$  \\

            0  &  0  &  1  &  1  &  0  &  1  &   $\neg x$  \\

            1  &  1  &  0  &  0  &  0  &  1  &   $\neg y$  \\

            0  &  0  &  1  &  1  &  1  &  1  &   $-x$  \\

            1  &  1  &  0  &  0  &  1  &  1  &   $-y$  \\

            0  &  1  &  1  &  1  &  1  &  1  &   $x + 1$  \\

            1  &  1  &  0  &  1  &  1  &  1  &   $y + 1$  \\

            0  &  0  &  1  &  1  &  1  &  0  &   $x - 1$  \\

            1  &  1  &  0  &  0  &  1  &  0  &   $y - 1$  \\

            0  &  0  &  0  &  0  &  1  &  0  &   $x + y$  \\

            0  &  1  &  0  &  0  &  1  &  1  &   $x - y$  \\

            0  &  0  &  0  &  1  &  1  &  1  &   $y - x$  \\

            0  &  0  &  0  &  0  &  0  &  0  &   $x \wedge y$  \\

            0  &  1  &  0  &  1  &  0  &  1  &   $x \vee y$  \\

            \bottomrule
        \end{tabular}
    \end{center}
    \label{tab:alu-functions}
    \caption{Functions that the ALU can calculate, given different settings of the control bits}
\end{table}
