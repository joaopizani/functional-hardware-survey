\section{Conclusions}
\label{sec:conclusions}

    The models and test cases that we developed, along with the verification we performed, gave us
    a better understanding of how the hardware \acp{EDSL} Lava, ForSyDe and Coquet compare to each other
    \emph{from the point of view of a hardware designer}. This practical experience, combined with
    knowledge of the ``inner workings'' of each \ac{EDSL} and their host language, allowed for an
    informed discussion of each language's strong points and weaknesses. The most significant
    findings of this practical evaluation, categorized by evaluated aspect, are summarized here.

    \paragraph{Depth of embedding}
        None of the three evaluated \acp{EDSL} lie at the \emph{extremes} of embedding depth. Lava can be
        said to be \emph{deeply embedded}, however, its \texttt{Signal} datatype collaborates with
        the host language runtime so that \emph{cyclic structures} in circuits can be modeled as
        \emph{recursion in the host language}. ForSyDe has \emph{both} deep and shallow modeling
        capabilities, even though we only studied the deep model. In fact, ForSyDe's hackage
        page\cite{website:forsyde-hackage} promises a future version in which deep and shallow
        modeling constructs will be in different packages. Coquet has the ``deepest'' modeling of
        all studied \acp{EDSL}, and \emph{avoids} the issue of \emph{observable sharing} by not allowing
        variable binding constructs, and having circuits connect to each other only through
        combinators.

    \paragraph{Simulation}
        Simulation can be performed in all studied \acp{EDSL}. In Lava, \emph{automated} test cases
        (in which the simulation output is compared with an \emph{expected} combination) are not
        possible due to the way in which the observable sharing issue is handled. Coquet has
        simulation built into the library as one of several \emph{example interpretations} for
        circuits, and it works just as well as in the other \acp{EDSL}, with the only shortcoming
        that simulation of \emph{sequential} circuits is currently \emph{not possible}.

    \paragraph{Verification}
        ForSyDe offers no capabilities for formal verification whatsoever, while Lava and Coquet
        each do, but in different ways. Lava can perform the verification of so-called \emph{safety
        properties} for circuits of a fixed size -- it does this by transforming the circuit model
        into a CNF (\emph{conjunctive normal form}) logical formula which is fed into a
        satisfiability solver. Coquet takes a different approach and offers some tools to help the
        user perform \emph{interactive theorem proving} for circuit correctness. One can say that
        Coquet does \emph{more than verification}, as with Coquet we can prove the correctness of
        whole families of parameterized circuits by induction.

    \paragraph{Genericity}
        In Lava the modeling of generic circuits is made very easy, and any parameter to a circuit
        definition which is not of type \texttt{Signal T} is considered a \emph{parameter} instead
        of a circuit input, and specific instances of these generic circuits can then be simulated
        or synthesized. In Coquet a similar approach is taken, allowing the user to prove \emph{by
        induction on the parameter} the correctness of the whole family of circuits.  ForSyDe is the
        \ac{EDSL} with the least opportunity for generalization: the only thing we can do is to have
        fixed-length bit vectors or fixed-size integers as inputs, and these are fixed at
        \emph{Haskell compilation time}.

    \paragraph{Tool integration}
        Lava can generate VHDL netlists of circuit models that satisfy some requirements and can
        also generate CNF formulas for a SAT solver. ForSyDe can output its circuits in VHDL and
        also generate \emph{graph files}, which can be formatted and used for circuit visualization.
        Coquet is disadvantaged when it comes to tool integration: it currently has no support for
        exporting circuits in some industry-standard format, even though one of the examples in the
        distribution is a gate-count, so netlist generation should be possible in the same
        framework.

    \paragraph{Extensibility}
        ForSyDe and Coquet offer both good capabilities for extensibility: in both \acp{EDSL} the
        designer can make circuits operate over user-defined types. The big advantage of ForSyDe is
        its usage of Template Haskell and GHC's \emph{deriving} mechanism to generate VHDL
        corresponding to the user-defined types. Lava offers little to no extensibility, and only
        circuits operating on booleans or integers can be modeled in the current version of Chalmers
        Lava.

