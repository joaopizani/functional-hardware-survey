\documentclass[a4paper]{article}

%% Unicode support
\usepackage[utf8x]{inputenc}
\usepackage{ucs}
\usepackage{autofe}

\usepackage{times}
\usepackage[T1]{fontenc}
\usepackage{a4}
\usepackage{epstopdf}
\usepackage{graphicx}
\usepackage{mathtools}
\usepackage{amssymb}
\usepackage{hyperref}
\usepackage{color}


%% Code
\usepackage{inconsolata}
\usepackage{minted}
\usepackage{fancyvrb}
\usemintedstyle{default}
\newminted{haskell}{gobble=4,linenos,mathescape,fontfamily=tt,fontsize=\footnotesize,xleftmargin=\parindent}



%% Metainformation
%% PDF stuff
\usepackage{datetime}
\usepackage{ifpdf}
\ifpdf
\pdfinfo{
    /Author (Joao Paulo Pizani Flor)
    /Title (A comparison of functional Embedded Domain-Specific Languages for hardware)
    /Keywords (EDSL, HDL, Hardware Description, Functional Programming, Haskell, Coq, Lava, Coquet)
    /CreationDate (D:\pdfdate)
}
\fi

\title{A comparison of functional Embedded Domain-Specific Languages for hardware}

\date{\today}

\author
{
    João Paulo Pizani Flor \\
    Department of Information and Computing Sciences, \\
    Utrecht University - The Netherlands \\
    e-mail: j.p.pizaniflor@students.uu.nl
}



%% The document itself
\begin{document}
    \maketitle

    \section{Introduction}
    \label{sec:intro}
        This report describes the solution to one of the practical assignments (project nr. 2)
        of the master course "Program Verification" at Utrecht University, taught in the 3rd period
        of the academic year 2012/2013.

    \section{Modelling the algorithm itself}
    \label{sec:algorithm}
        The system we were asked to model consisted of a set of processes $ Sys_{n} $:
        \[
            Sys_{n} = \{ \text{PMIN}(n-1, 0) \} \cup \{ \text{PMIN}(i, i+1) \mid i < n-1 \}
        \]


    \section{Formalizing the specifications to be verified}
    \label{sec:specifications}

        Given the ``basic'' definitions above, now we give HOL definitions formalizing the
        specifications to be verified. First of all, there will be three categories of HOL
        definitions, one category for each of the approaches we considered (set, function, list).

        \subsection{Specifications using the set model}
        \label{subsec:spec_set}
            When using the set model, the specifications correspond to the following HOL
            definitions:


\end{document}
